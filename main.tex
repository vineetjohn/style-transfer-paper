\documentclass[letterpaper]{article} %DO NOT CHANGE THIS
\usepackage{aaai19}  %Required
\usepackage{times}  %Required
\usepackage{helvet}  %Required
\usepackage{courier}  %Required
\usepackage{url}  %Required
\usepackage{graphicx}  %Required

\usepackage{amsmath,amsfonts,amssymb}
\usepackage{paralist}
\usepackage{color,xcolor}
\usepackage{bm}
\usepackage{multirow}
\usepackage{makecell}
\usepackage{caption}
\usepackage[linesnumbered,lined]{algorithm2e}
\usepackage{todonotes}

\frenchspacing  %Required
\setlength{\pdfpagewidth}{8.5in}  %Required
\setlength{\pdfpageheight}{11in}  %Required

%PDF Info Is Required:
%\pdfinfo{
%	/Title (Disentangled Representation Learning for Text Style Transfer)
%	/Author (Vineet John, Lili Mou, Hareesh Bahuleyan, Olga Vechtomova)
%}


% bold x for equations
\newcommand{\rmx}{\mathrm x} 
% used for table headers
\newcommand{\tabh}[1]{\multicolumn{1}{c|}{\textbf{#1}}}  
% used for the top left cell in a table
\newcommand{\tabc}[2]{\multicolumn{1}{|c||}{\multirow{#1}{*}{\textbf{#2}}}} 
% used for denoting the loss symbol with a subscript
\newcommand{\loss}[1]{J_{\text{#1}}}
% used for denoting the lambda symbol with a subscript
\newcommand{\hyp}[1]{\lambda_{\text{#1}}}
% used for denoting the theta symbol with a subscript
\newcommand{\nnweight}[1]{\bm{\theta_{\text{#1}}}}
% used for denoting the weight symbol (w) with a subscript
\newcommand{\weight}[1]{W_{\text{#1}}}
% used for denoting the bias symbol (b) with a subscript
\newcommand{\bias}[1]{\bm{b_{\text{#1}}}}
% custom author-year citation
\newcommand{\citeay}[1]{\citeauthor{#1} \shortcite{#1}}

\title{Disentangled Representation Learning for Non-Parallel Text Style Transfer}

%\author{
%	Vineet John \\
%	University of Waterloo \\
%	{\tt vineet.john@uwaterloo.ca} \\
%	\And
%	Lili Mou \\
%	AdeptMind Research \\
%	{\tt doublepower.mou@gmail.com}\\{\tt lili@adeptmind.ai} \\
%	\AND
%	Hareesh Bahuleyan \\
%	University of Waterloo \\
%	{\tt hpallika@uwaterloo.ca} \\
%	\And
%	Olga Vechtomova \\
%	University of Waterloo \\
%	{\tt ovechtom@uwaterloo.ca} \\
%}

\date{}
\begin{document}
\maketitle
\graphicspath{{images/}}

\begin{abstract}
	This paper tackles the problem of disentangling the latent variables of style and content in language models.
	We propose a simple yet effective approach, which incorporates auxiliary multi-task and adversarial objectives, for label prediction and bag-of-words prediction, respectively.
	We show, both qualitatively and quantitatively, that the style and content are indeed disentangled in the latent space.
	This disentangled latent representation learning method is applied to style transfer on non-parallel corpora.
	We achieve similar content preservation scores compared to previous state-of-the-art approaches, and significantly better style-transfer strength and language fluency scores.
	\footnote{
		Our code and data are publicly available at the following URL: \url{bit.ly/2wxps69}.
		The materials are properly anonymized during double-blind review.
	}
	% \footnote{\url{https://github.com/vineetjohn/linguistic-style-transfer}}.
\end{abstract}

% 


\section{Introduction}

The neural network has been a successful learning machine during the past decade due to its highly expressive modeling capability, which is a consequence of multiple layers of non-linear transformations of input features.
Such transformations, however, make intermediate features ``latent'', in the sense that, they do not have explicit meaning and are not interpretable.
Therefore, neural networks are usually treated as black-box machinery.

Disentangling the latent space of neural networks has become an increasingly important research topic.
In the image domain, for example, \citeay{chen2016infogan} use adversarial and information maximization objectives to produce interpretable latent representations that can be tweaked to adjust writing style for handwritten digits, as well as lighting and orientation for face models.
\citeay{mathieu2016disentangling} utilize a convolutional autoencoder to achieve the same objective.
However, this problem is not well explored in natural language processing.

In this paper, we address the problem of disentangling the latent space of neural networks for text generation.
Our model is built on an autoencoder that encodes a sentence to the latent space (vector representation) by learning to reconstruct the sentence itself.
We would like the latent space to be disentangled with respect to different features, namely, \textit{style} and \textit{content} in our task.

To accomplish this, we propose a simple yet effective approach that combines multi-task and adversarial objectives.
We artificially divide the latent representation into two parts: the style space and content space. In this work, we consider the sentiment of a sentence as the style.
We design auxiliary losses, enforcing our model to separate style and content latent spaces.
In particular, the multi-task loss operates on a latent space to ensure that the space does contain the information we wish to encode.
The adversarial loss, on the contrary, minimizes the predictability of information that should not be contained in that space.
In previous work, researchers typically work with the style, or specifically, sentiment space~\cite{hu2017toward,shen2017style,fu2018style,zhao2018adversarially}, but simply ignore the content space, as it is hard to formalize what ``content'' actually refers to.

In our paper, we propose to approximate the content information by bag-of-words (BoW) features, where we focus on style-neutral, non-stop words.
Along with traditional style-oriented auxiliary losses, our BoW multi-task loss and BoW adversarial loss make the style and content spaces much more disentangled from each other.

The disentangled latent space can be directly used for text style-transfer~\cite{hu2017toward,shen2017style}, where a model transforms a given sentence to a new sentence with the same content but a different style.
Since it is difficult to obtain sentence pairs with the same content and differing styles (i.e. parallel corpora), we follow the setting where we train our model on a non-parallel but style-labeled corpora. We call this \textit{non-parallel text style transfer}.
To accomplish this, we train an autoencoder with disentangled latent space during training.
For style-transfer inference, we simply use the autoencoder to encode the content vector of a sentence, but ignore its encoded style vector.
We then infer from the training data an empirical embedding of the style that we would like to transfer.
The encoded content vector and the empirically-inferred style vector are concatenated and fed to the decoder.
This grafting technique enables us to obtain a new sentence similar in content to the input sentence, but with a different style.

We conducted experiments on two customer review datasets.
Both qualitative and quantitative results show that both the style and content spaces are indeed disentangled well.
In the style-transfer evaluation, we achieve significantly better style-transfer strength and language fluency compared to previous results, while obtaining comparable content preservation scores.
Ablation tests also show that the auxiliary losses can be combined well, each playing its own role in disentangling the latent space.


\section{Related Work}

Disentangling neural networks' latent space has been explored in the image processing domain in the recent years, and researchers have successfully disentangled rotation features, color features, etc.~\cite{chen2016infogan,luan2017deep}.
Some image characteristics (e.g., artistic style) can be captured well by certain statistics \cite{gatys2016image}.
In other work, researchers adopt data augmentation techniques to learn a disentangled latent space~\cite{kulkarni2015deep,champandard2016semantic}.

In natural language processing, the definition of ``style'' itself is vague, and as a convenient starting point, NLP researchers often treat sentiment as a salient style of text.
\citeay{hu2017toward} manage to control the sentiment by using discriminators to reconstruct sentiment and content from generated sentences.
However, there is no evidence that the latent space would be disentangled by this reconstruction.
\citeay{shen2017style} use a pair of adversarial discriminators to align the recurrent hidden decoder states of original and style-transferred sentences, for a given style.
\citeay{fu2018style} propose two approaches: using a style embedding matrix, and using style-specific decoders for style-transfer.
They apply an adversarial loss on the encoded space to discourage encoding style in the latent space of an autoencoding model. However, all the above approaches only deal with the style information and simply ignore the content part.

\citeay{zhao2018adversarially} extend the multi-decoder approach and use a Wasserstein-distance penalty to align content representations of sentences with different styles. However, the Wasserstein penalty is applied to  empirical samples from the data distribution, and is more indirect than our BoW-based auxiliary losses.
Recently, \citeay{rao2018dear} treat the formality of writing as a style, and create a parallel corpus for style transfer with sequence-to-sequence models.
This is beyond the scope of our paper, as we focus on non-parallel text style transfer.

Our paper differs from previous work in that both our style space and content space are encoded from the input.
We apply several auxiliary losses to ensure that each space encodes and only encodes the desired information.
Such disentanglement of latent space has its own research interest in the deep learning community.
The disentangled representation can then be directly applied to non-parallel text style-transfer tasks, as in the aforementioned studies.


\section{Approach}

In this section, we describe our approach in detail, shown in Figure~\ref{fig:overview}.
Our model is built upon an autoencoder with a sequence-to-sequence neural network~\cite{sutskever2014sequence}.
Then, we introduce the multi-task and adversarial auxiliary losses for both style and content spaces.
Finally, we present our approach to transfer style in the context of natural language generation.

\begin{figure}[!t]
	\centering
	\includegraphics[width=.9\linewidth]{model-overview}
	\caption{Overview of our approach.}
	\label{fig:overview}
\end{figure}

\subsection{Autoencoder} \label{ssec:seq2seq-autoencoder}

An autoencoder encodes an input to a latent vector space, from which it reconstructs the input itself.
The latent vector space is usually of much smaller dimensionality than input data, and the autoencoder learns salient and compact representations of data during the reconstruction process.
This serves as our primary learning objective.
Besides, we also use the autoencoder for text generation in the style-transfer application.

Let $\rmx=(x_1, x_2, \cdots x_n)$ be an input sequence with $n$ tokens.
The encoder recurrent neural network (RNN) with gated recurrent units (GRU) \cite{cho2014learning} encodes $\rm x$ and obtains a hidden state $\bm h$.
Then a decoder RNN generates a sentence, which ideally should be $\rmx$ itself.
Suppose at a time step $t$, the decoder RNN predicts the word $x_t$ with probability $p(x_t|\bm h, x_1\cdots x_{t-1})$. Then the autoencoder is trained with a sequence-aggregated cross-entropy loss, given by
\begin{equation}
	\loss{AE}(\nnweight{E},\nnweight{D})= -\sum_{t=1}^n \log p(x_t|\bm h, x_1\cdots x_{t-1})
\end{equation}
where $\nnweight{E}$ and $\nnweight{D}$ are the parameters of the encoder and decoder, respectively.\footnote{For brevity, we only present the loss for a particular data point (i.e., a sentence) throughout the paper. The total loss sums over all data points, but is implemented with mini-batches.} Both the encoder and decoder are deterministic functions in the original autoencoder model~\cite{rumelhart1985learning}, and thus we call it a \textit{deterministic autoencoder} (DAE).

\subsubsection{Variational Autoencoder.}

In addition to the deterministic auto-encoding objective, we also tried a variational autoencoder (VAE) \cite{kingma2013auto}, which imposes a probabilistic distribution to the latent vector. The Kullback-Leibler (KL) divergence \cite{kullback1951information} penalty is added to the loss function to regularize the latent space. The decoder reconstructs data based on the sampled latent vector from its posterior distribution.

Formally, the autoencoding loss in the VAE is
\begin{align}
	\loss{AE}(\nnweight{E}, \nnweight{D}) = & - \mathbb{E}_{q_{E}(\bm h|\rmx)} [\log p(\rmx|\bm h)]  \nonumber \\
	                                        & + \hyp{kl}\operatorname{KL}(q_{E}(\bm h|\rmx)\|p(\bm h))
\end{align}
where $\hyp{kl}$ is the hyperparameter balancing the reconstruction loss and the KL term. $p(\bm h)$ is the prior, set to the standard normal distribution $\mathcal{N}(\bm 0,\mathrm I)$. The motivation for using VAE as opposed to DAE is  that the reconstruction is based on the samples of the posterior, which theoretically populates the neighborhood and thus smooths the latent space. \citeay{bowman2016generating} show that VAE generates more fluent sentences than DAE.

Besides the above autoencoding loss, we design several auxiliary losses to disentangle the latent space. In particular, we hope that $\bm h$ can be separated into two spaces $\bm s$ and $\bm c$, representing style and content respectively, i.e., $\bm h = [\bm s ; \bm c]$, where $[\cdot;\cdot]$ denotes concatenation.
This is accomplished by the auxiliary losses described in the rest of this section.


\subsection{Style-Oriented Losses}

We first design auxiliary losses that ensure the style information is contained in the style space $\bm s$.
This involves a multi-task loss that ensures $\bm s$ is discriminative for the style, as well as an adversarial loss that ensures $\bm c$ is not discriminative for the style.

\subsubsection{Multi-Task Loss for Style.} \label{ssec:multitask-style-objective}
Although the corpus we use is non-parallel, we assume that each sentence is labeled with its style. In particular, we treat the sentiment as the style of interest, following previous work~\cite{hu2017toward,shen2017style,fu2018style,zhao2018adversarially}, and each sentence is labeled with a binary sentiment tag (positive or negative).

We build a classifier on the style space that predicts the style label. Formally, a two-way softmax layer (equivalent to logistic regression) is applied to the style vector $\bm s$, given by
\begin{equation} \label{eqn:class-pred}
	\bm y_s = \operatorname{softmax}(\weight{mul(s)} \bm s + \bias{mul(s)})
\end{equation}
where $\nnweight{mul(s)}=[\weight{mul(s)}; \bias{mul(s)}]$ are parameters for multi-task learning of style, and $\bm y_s$ is the output of softmax layer.

The classifier is trained with a simple cross-entropy loss against the ground truth distribution $t_s(\cdot)$, given by
\begin{equation} \label{eqn:style-multi-task-loss}
	\loss{mul(s)}(\nnweight{E};\nnweight{mul(s)}) = - \sum_{l\in\text{labels}} t_s(l)\log y_s(l)
\end{equation}
where $\nnweight{E}$ are the encoder's parameters.

We train the style classifier at the same time as the autoencoding loss.
Thus, this could be viewed as \textit{multi-task} learning, incentivizing the entire model to not only decode the sentence, but also predict its sentiment from the style vector $\bm  s$.
We denote it by ``mul(s)''.
Similar multi-task losses are used in previous work for sequence-to-sequence learning \cite{luong2015multi}, sentence representation learning \cite{jernite2017discourse} and sentiment analysis \cite{balikas2017multitask}, among others.


\subsubsection{Adversarial Loss for Style.}
\label{ssec:adversarial-style-objective}

The above multi-task loss only ensures that the style space contains style information.
However, the content space might also contain style information, which is undesirable for disentanglement and style transfer.

We thus apply an adversarial loss to disentangle the content space from style information.
The idea is to first introduce a classifier, called an \textit{adversary}, that deliberately discriminates the true style label using the content vector $\bm c$.
Then the encoder is trained to learn a content vector space, from which its adversary cannot predict style information.

Concretely, the adversarial discriminator and its training objective have a similar form as Equations~\ref{eqn:class-pred} and~\ref{eqn:style-multi-task-loss}, but with different input and parameters, given by
\begin{align}
	\label{eqn:adv-disc-loss}
	\bm y_s                          & = \operatorname{softmax}(\weight{dis(s)} \bm c + \bias{dis(s)}) \\
	\loss{dis(s)}(\nnweight{dis(s)}) & = - \sum_{l\in\text{labels}} t_c(l)\log y_s(l)
\end{align}
where $\nnweight{dis(s)}=[\weight{dis(s)}; \bias{dis(s)}]$ are the parameters of the adversary.

It should be emphasized that, for the adversary, the gradients are not propagated back to the autoencoder, i.e. the variables in $\bm c$ are treated as shallow features. Therefore, we view $\loss{dis(s)}$ as a function of $\nnweight{dis(s)}$ only, whereas $\loss{mul(s)}$ is a function of both $\nnweight{E}$ and $\nnweight{mul(s)}$.

Having trained an adversary, we would like the autoencoder to be tuned in such an \textit{ad hoc} fashion, that $\bm c$ is not discriminative for style.
In existing literature, there could be different approaches, for example, maximizing the adversary's loss~\cite{shen2017style,zhao2018adversarially} or penalizing the entropy of the adversary's prediction~\cite{fu2018style}.
In our work, we adopt the latter, as it can be easily extended to multi-category classification, used for the content-oriented losses of our approach. Formally, the adversarial objective for the style is to maximize
\begin{equation} \label{eqn:advs}
	\loss{adv(s)}(\nnweight{E})=\mathcal{H}(\bm y_s|\bm c; \nnweight{dis(s)})
\end{equation}
where $\mathcal{H}(\bm p)=-\sum_{i\in\text{labels}}p_i\log p_i$ and $\bm y_s$ is the predicted distribution over the style labels. Here, $\loss{adv(s)}$ is maximized with respect to the encoder. It is viewed as a function of $\nnweight{E}$, and we fix $\nnweight{dis(s)}$.

While adversarial loss has been explored in previous style-transfer papers~\cite{shen2017style,fu2018style,zhao2018adversarially}, it has not been combined with the multi-task loss. As we shall show in our experiments, combining these two losses is promisingly effective, achieving better text style transfer performance than a variety of previous state-of-the-art methods.

\subsection{Content-Oriented Losses}

The above style-oriented losses only regularize style information, but they do not impose any constraint on where the content information should be encoded. This also happens in previous work~\cite{hu2017toward,shen2017style,fu2018style,zhao2018adversarially}.
Although the style space is usually much smaller than the content space, it is unrealistic to expect that the content would not flow into the style space because of its limited capacity. Therefore, we need to design content-oriented auxiliary losses to regularize the content information.

Inspired by the above combination of multi-task and adversarial losses, we apply the same idea to the content space. However, it is hard to define what ``content'' actually refers to.

To this end, we propose that the content information be approximated using bag-of-words (BoW) features.
The input sentences $\rmx$ are represented as vectors of the same size as the corpus vocabulary, with each index of the vector denoting the frequency of a word's presence in the sentence.
For a sentence $s$ with $N$ tokens, a word $w_*$'s BoW probability is given by
$t_c(w_*)=\frac{\sum_{i=1}^{N}{\mathbb{I}\{w_i = w_*\}}}{N}$
where $t_c(\cdot)$ denotes the target distribution of content, and $\mathbb{I}$ is an indicator function.

However, we exclude stopwords and style-specific words. For sentiment style transfer, we exclude sentiment words from a curated lexicon \cite{hu2004mining}.


\subsubsection{Multi-Task Loss for Content.} \label{ssec:multitask-content-objective}

Similar to the style-oriented loss, the multi-task loss for content, denoted as ``mul(c)'', ensures that the content space $\bm c$ contains content information, i.e., BoW features.

We introduce a softmax classifier over the BoW vocabulary
\begin{equation} \label{eqn:bow-pred}
	\bm y_c = \operatorname{softmax}({\weight{mul(c)}} \bm c + \bias{mul(c)})
\end{equation}
where $\nnweight{mul(c)}=[\weight{mul(c)}; \bias{mul(c)}]$ are the classifier's parameters, and $\bm y_c$ is the predicted BoW distribution.

The training objective is a cross-entropy loss against the ground truth distribution $t_c(\cdot)$, given by
\begin{equation}\label{eqn:content-multi-task-loss}
	\loss{mul(c)}(\nnweight{E};\nnweight{mul(c)}) = - \sum_{w\in\text{words}} t_c(w)\log y_c(w)
\end{equation}
where the optimization is performed with both encoder parameters $\nnweight{E}$ and the multi-task classifier $\nnweight{mul(c)}$.

Notice that although the target distribution is not one-hot as in most classification problems, the cross-entropy loss (Equation~\ref{eqn:content-multi-task-loss}) has the same form. It is also interesting that, at the first glance, the multi-task loss for content is redundant with the autoencoding loss, when in fact, it is not. The multi-task loss only considers content words, which do not include stopwords and sentiment words, and is only applied to the content space $\bm c$. Together with the style space $\bm s$, the autoencoding loss requires the model to generate the entire sentence. This ensures that the content information is indeed captured in the content space.

\subsubsection{Adversarial Loss for Content.} \label{ssec:adversarial-content-objective}

To ensure that the style space does not contain content information, we design our last auxiliary loss, the adversarial loss for content, denoted as ``adv(c)''.

We build an adversary, a softmax classifier on the style space to predict BoW features, approximating content information, given by
\begin{align}
	\label{eqn:adv-bow-disc-loss}
	\bm y_c                          & = \operatorname{softmax}({\weight{dis(c)}}^\top \bm s + \bias{dis(c)}) \\
	\loss{dis(c)}(\nnweight{dis(c)}) & = - \sum_{w\in\text{words}} t_c(w)\log y_c(w)
\end{align}
where $\nnweight{dis(c)}=[\weight{dis(c)}; \bias{dis(c)}]$ are the classifier's parameters for BoW prediction.

The adversarial loss for the model is to maximize the entropy of the discriminator
\begin{equation}
	\loss{adv(c)}(\nnweight{E}) = \mathcal{H}(\bm y_c | \bm s; \nnweight{dis(c)})
\end{equation}
Again, $\loss{dis(c)}$ is trained with respect to the discriminator's parameters $\nnweight{dis(c)}$, whereas $\loss{adv(c)}$ is trained with respect to $\nnweight{E}$, similar to the adversarial loss for style.




\subsection{Training Process}

The overall loss $\loss{ovr}$ for the autoencoder comprises several terms: the reconstruction objective, the multi-task objectives for style and content, and the adversarial objectives for style and content:
\begin{align}
	\loss{ovr} = & \loss{AE}(\nnweight{E}, \nnweight{D})  \nonumber                                                                   \\
	             & + \hyp{mul(s)} \loss{mul(s)} (\nnweight{E},\nnweight{mul(s)}) - \hyp{adv(s)} \loss{adv(s)}(\nnweight{E}) \nonumber \\
	             & + \hyp{mul(c)} \loss{mul(c)} (\nnweight{E},\nnweight{mul(c)}) - \hyp{adv(c)} \loss{adv(c)}(\nnweight{E})
\end{align}
where $\lambda$'s are the hyperparameters that balance the autoencoding loss and these auxiliary losses.

To put it all together, the model training involves an alternation of optimizing discriminator losses $\loss{dis(s)}$ and $\loss{dis(c)}$, and the model's own loss $\loss{ovr}$, shown in Algorithm~\ref{alg:training-process}.


\begin{algorithm}[!t]
	\ForEach{mini-batch}{
		minimize $\loss{dis(s)}(\nnweight{dis(s)})$ w.r.t. $\nnweight{dis(s)}$\;
		minimize $\loss{dis(c)}(\nnweight{dis(c)})$ w.r.t. $\nnweight{dis(c)}$\;
		minimize $\loss{ovr}$ w.r.t. $\nnweight{E}, \nnweight{D}, \nnweight{mul(s)}, \nnweight{mul(c)}$\;
	}
	\caption{Training process.}
	\label{alg:training-process}
\end{algorithm}


\subsection{Generating Style-Transferred Sentences} \label{ssec:sentence-generation}

A direct application of our disentangled latent space is style-transfer for natural language generation.
For example, we can generate a sentence with generally the same meaning (content) but a different style (e.g. sentiment).

Let $\rmx^*$ be an input sentence with $\bm s^*$ and $\bm c^*$ being the encoded, disentangled style and content vectors, respectively.
If we would like to transfer its content to a different style, we compute an empirical estimate of the target style's vector $\hat{\bm s}$ using
\begin{equation*}
	\hat{\bm s}=\frac{\sum_{i\in\text{target style}}\bm s_i}{\text{\# target style samples}}
\end{equation*}
The inferred target style $\hat{\bm s}$ is concatenated with the encoded content $\bm c^*$ for decoding style-transferred sentences, as shown in Figure~\ref{fig:overview}b.


\section{Experiments}

\subsection{Datasets}

We conducted experiments on two datasets, Yelp and Amazon reviews.
Both of these datasets comprise sentences accompanied by binary sentiment labels (positive, negative). They are used to train disentangled latent space as well as to evaluate sentiment transfer.

\subsubsection{Yelp Service Reviews.}
We use a Yelp review dataset, following previous work \cite{shen2017style,zhao2018adversarially}.
\footnote{\url{https://github.com/shentianxiao/language-style-transfer}}
It contains 444,101, 63,483 and 126,670 sentences for train, validation, and test, respectively.
The maximum sentence length is 15, and the vocabulary size is approximately 9,200.

\subsubsection{Amazon Product Reviews.}
We also used an Amazon review dataset, following previous work \cite{fu2018style}.
\footnote{\url{https://github.com/fuzhenxin/text_style_transfer}}
It contains 559,142, 2,000 and 2,000 sentences for train, validation, and test, respectively.
The maximum sentence length is 20, and the vocabulary size is approximately 58,000.


\subsection{Experiment Settings}
We used the Adam optimizer \cite{kingma2014adam} for the autoencoder and the RMSProp optimizer \cite{tieleman2012lecture} for the discriminators, each with an initial learning rate of $10^{-3}$.
Our model is trained for 20 epochs, by which time it has converged.
The word embedding layer was initialized by word2vec \cite{mikolov2013distributed} trained on respective training sets.
Both the autoencoder and the discriminators are trained once per epoch with $\hyp{mul(s)} = 10$, $\hyp{mul(c)} = 3$, $\hyp{adv(s)} = 1$ and $\hyp{adv(c)} = 0.03$.
These hyperparameters were tuned by performing a log-scale grid search within two orders of magnitude around the default value $1$, and choosing those that yielded the best validation results.
The recurrent unit size is $256$, the style vector size is $8$, and the content vector size is $128$.
We append the latent vector $\bm h$ to the hidden state at every time step of the decoder.
For the VAE model, we enforce the KL-divergence penalty on both the style and content posterior distributions, using $\hyp{kl(s)}$ and $\hyp{kl(c)}$, respectively.
We set $\hyp{kl(s)} = 0.03$ and $\hyp{kl(c)} = 0.03$ and use the $\operatorname{sigmoid}$ KL-weight annealing schedule following \citeay{bahuleyan2018probabilistic}. They were tuned along with the other hyperparameters of the disentangled model.




\subsection{Experiment I: Disentangling Latent Space}

First, we analyze how the style (sentiment) and content of the latent space are disentangled.
We train classifiers on the different latent spaces, and report their inference-time classification accuracies in Tables~\ref{tab:classification-yelp} and~\ref{tab:classification-amazon} for the Yelp and Amazon datasets, respectively.

We see that the 128-dimensional content vector $\bm c$ is not particularly discriminative for style.
It achieves accuracies slightly better than majority guess.
However, the 8-dimensional style vector $\bm s$, despite its low dimensionality, achieves significantly higher style classification accuracy.
When combining content and style vectors, we achieve no further improvement.
These results verify the effectiveness of our disentangling approach, as the style space contains style information, whereas the content space does not.

We show t-SNE plots of both the deterministic autoencoder (DAE) and the variational autoencoder (VAE) models in Figure~\ref{fig:tsne-plots}.
As seen, sentences with different styles are noticeably separated in a clean manner in the style space (LHS), but are indistinguishable in the content space (RHS).
It is also evident that the latent space learned by the variational autoencoder is considerably smoother and continuous compared to the one learned by the deterministic autoencoder.
\begin{table}[!t]
	\centering
	\begin{tabular}{| l || r | r |}
		\hline
		\tabc{1}{Latent Space}           & \tabh{DAE}                 & \tabh{VAE} \\
		\hline \hline
		Majority guess                   & \multicolumn{2}{c|}{0.602}              \\
		\hline
		Content space  ($\bm c$)         & 0.658                      & 0.697      \\ \hline
		Style space ($\bm s$)            & 0.974                      & 0.974      \\ \hline
		Complete space ($[\bm s;\bm c]$) & 0.974                      & 0.974      \\
		\hline
	\end{tabular}
	\caption{Classification accuracy on latent spaces (Yelp).}
	\label{tab:classification-yelp}
	\begin{tabular}{| l || r | r |}
		\hline
		\tabc{1}{Latent Space}           & \tabh{DAE}                 & \tabh{VAE} \\
		\hline \hline
		Majority guess                   & \multicolumn{2}{c|}{0.512}              \\
		\hline
		Content space  ($\bm c$)         & 0.675                      & 0.693      \\ \hline
		Style space ($\bm s$)            & 0.821                      & 0.810      \\ \hline
		Complete space ($[\bm s;\bm c]$) & 0.819                      & 0.810      \\
		\hline
	\end{tabular}
	\caption{Classification accuracy on latent spaces (Amazon).}
	\label{tab:classification-amazon}
\end{table}
\begin{figure}[!t]
	\centering
	\includegraphics[width=\linewidth]{latent-spaces}
	\caption{t-SNE plots of our disentangled models.}
	\label{fig:tsne-plots}
\end{figure}





\begin{table*}[!t]
	\centering
    \resizebox{\textwidth}{!}{
	\begin{tabular}{|l||c|c|c|c||c|c|c|c| }
		\hline
		\tabc{3}{Model}    & \multicolumn{4}{c||}{\textbf{Yelp Dataset}} & \multicolumn{4}{c|}{\textbf{Amazon Dataset}}\\
        \cline{2-9}
      & \textbf{Transfer} & \textbf{Cosine}     & \textbf{Word}    & \textbf{Language} & \textbf{Transfer} & \textbf{Cosine}     & \textbf{Word}    & \textbf{Language}\\
		                                           &\textbf{Accuracy} &  \textbf{Similarity} & \textbf{Overlap} & \textbf{Fluency} &\textbf{Accuracy} & \textbf{Similarity} & \textbf{Overlap} & \textbf{Fluency}  \\
		\hline
		\hline
        Style-Embedding \cite{fu2018style}         & 0.182           & \textbf{0.959}             & \textbf{0.666}          & -16.17         & 0.417           & \textbf{0.933}             & \textbf{0.359}          & -28.13    \\
        \hline		\hline
		Cross-Alignment \cite{shen2017style}       & 0.809           & 0.892             & 0.209          & -23.39  & 0.606           & 0.893             & 0.024          & -26.31          \\
		\hline
		Multi-Decoder \cite{zhao2018adversarially} & 0.835           & 0.883             & 0.272          & -20.95    & 0.552           & \textit{0.926}             & 0.169          & -34.70        \\
		\hline
		Ours (DAE)                                 & 0.883           & \textit{0.915}             & \textit{0.549}          & -10.17     & 0.720           & 0.921             & \textit{0.354}          & -24.74       \\
		\hline
		Ours (VAE)                                 & \textbf{0.934}           & {0.904}             & {0.473}          & \textbf{-9.84}         & \textbf{0.822}           & 0.900             & 0.196          & \textbf{-21.70}         \\
		\hline
	\end{tabular}}
	\caption{Performance of non-parallel text style transfer. The style-embedding approach achieves poor transfer accuracy, and should not be considered as an effective4 style-transfer model. Despite this, our model outperforms other previous methods in terms of all aspects (transfer length, content preserving, and language fluency).}
	\label{tab:yelp-comparison-previous}
\end{table*}


\subsection{Experiment II: Non-Parallel Text Style Transfer}
We then evaluate the performance of style transfer with our disentangled latent space.

\subsubsection{Metrics.} We shall evaluate competing models by (1) how the sentiment is transferred,  and (2) how the content is preserved. The evaluation of sentence generation is usually hard, and we adopt a few automatic metrics as well as human judgment.

$\bullet$ \textit{Style-Transfer Strength.} We follow most previous work~\cite{hu2017toward,shen2017style,fu2018style} and train a separate convolutional neural network (CNN)  to predict the sentiment of a sentence~\cite{kim2014convolutional}, which is then used to approximate the style transfer accuracy. In other words, we report the CNN classifier's accuracy on the style-transferred sentences, considering the target style to be the ground truth.

While the style classifier itself may not be perfect, it achieves a reasonable sentiment accuracy on the validation $97\%$ for Yelp and $82\%$ for the Amazon. Thus, it provides a quantitative way of evaluating the strength of style-transfer.

$\bullet$ \textit{Cosine Similarity.}
We followed \citeay{fu2018style} and computed a sentence embedding by concatenating the $\operatorname{min}$, $\operatorname{max}$, and $\operatorname{mean}$ of its constituent word embeddings (sentiment words are removed).
Then, the cosine similarity between the source and generated sentence embeddings is computed to evaluate how close they are in meaning, which more or less indicates how content is preserved.

$\bullet$ \textit{Word Overlap.} We find that the cosine similarity measure is not sensitive to content preserving, and we propose a simple yet effective measure that counts the unigram word overlap rate of the original sentence $\mathrm x$ and the style-transferred sentence $\mathrm y$, computed by $\operatorname{word-overlap} = \frac{count(w_{\mathrm x} \cap w_{\mathrm y})}{count(w_{\mathrm x} \cup w_{\mathrm y})}$.




$\bullet$ \textit{Language Fluency.}
We use a trigram Kneser-Ney (KL) smoothed language model \cite{kneser1995improved} as a quantitative and automated metric to evaluate the fluency of a sentence.
It estimates the empirical distribution of trigrams in a corpus, and computes the log-likelihood of a test sentence.
We train the language model on both Yelp and Amazon, and report the Kneser-Ney language model's log-likelihood. A smaller (more negative) number indicates a more fluent sentence.

$\bullet$  \textit{Manual Evaluation.} Despite the above automatic metrics, we also have human evaluation to further confirm the performance of our model. This was conducted on the Yelp dataset only, due to the large amount of human effort involved. We asked 6 human annotators to rate each sentence on a 1--5 Likert scale~\cite{stent2005evaluating} in terms of transfer strength, content similarity, and language quality. The human evaluation was conducted in a strictly blind fashion: samples obtained by all competing models are randomly shuffled, so that the annotator could not know which model generated a particular sentence.
\begin{table}[!t]
	\centering
    \resizebox{\linewidth}{!}{
	\begin{tabular}{| l || c | c | c | }
		\hline
		\tabc{2}{Model}                    & \tabh{Transfer} & \tabh{Content}      & \tabh{Language} \\
		                                   & \tabh{Strength} & \tabh{Preservation} & \tabh{Quality}  \\
		\hline
		\hline		\citeay{fu2018style}           & 1.67            & \textbf{3.84}       & 3.66            \\\hline\hline
		\citeay{shen2017style}         & 3.63            & 3.07                & 3.08            \\
		\hline
		\citeay{zhao2018adversarially} & 3.55            & 3.09                & 3.77            \\
		\hline
		Ours (DAE)                         & 3.67            & 3.64                & 4.19            \\
		\hline
		Ours (VAE)                         & \textbf{4.32}   & 3.73                & \textbf{4.48}   \\
		\hline
	\end{tabular}}
	\caption{Manual evaluation on the Yelp dataset.}
	\label{tab:manual-evaluation}
\end{table}

\begin{table*}[ht]
	\centering
	\scriptsize
	\begin{tabular}{| p{0.3\linewidth} || p{0.3\linewidth} | p{0.3\linewidth} |}
		\hline
		\tabc{2}{Original (Positive)}                                           & \tabh{DAE Transferred}                                                    & \tabh{VAE Transferred}                                     \\
		                                                                        & \tabh{(Negative)}                                                         & \tabh{(Negative)}                                          \\
		\hline
		\hline
		the food is excellent and the service is exceptional                    & the food was a bit bad but the staff was exceptional                      & the food was bland and i am not thrilled with this         \\
		\hline
		the waitresses are friendly and helpful                                 & the guys are rude and helpful                                             & the waitresses are rude and are lazy                       \\
		\hline
		the restaurant itself is romantic and quiet                             & the restaurant itself is awkward and quite crowded                        & the restaurant itself was dirty                            \\
		\hline
		great deal                                                              & horrible deal                                                             & no deal                                                    \\
		\hline
		both times i have eaten the lunch buffet and it was outstanding         & their burgers were decent but the eggs were not the consistency           & both times i have eaten here the food was mediocre at best \\
		\hline
		\hline
		\tabc{2}{Original (Negative)}                                           & \tabh{DAE Transferred}                                                    & \tabh{VAE Transferred}                                     \\
		                                                                        & \tabh{(Positive)}                                                         & \tabh{(Positive)}                                          \\
		\hline
		\hline
		the desserts were very bland                                            & the desserts were very good                                               & the desserts were very good                                \\
		\hline
		it was a bed of lettuce and spinach with some italian meats and cheeses & it was a beautiful setting and just had a large variety of german flavors & it was a huge assortment of flavors and italian food       \\
		\hline
		the people behind the counter were not friendly whatsoever              & the best selection behind the register and service presentation           & the people behind the counter is friendly caring           \\
		\hline
		the interior is old and generally falling apart                         & the decor is old and now perfectly                                        & the interior is old and noble                              \\
		\hline
		they are clueless                                                       & they are stoked                                                           & they are genuinely professionals                           \\
		\hline
	\end{tabular}
	\caption{Examples of style transfer sentence generation.}
	\label{tab:transfer-samples}
\end{table*}


\subsubsection{Results and Analysis.}

We compare our approach with previous state-of-the-art work in Table \ref{tab:yelp-comparison-previous}.
For baseline methods, we replicated the experiments with publicly available code released in \citeay{fu2018style}, \citeay{shen2017style}, and \citeay{zhao2018adversarially}.

We observe that the style embedding model \cite{fu2018style} performs poorly on the style-transfer objective,\footnote{It should be noted that the transfer accuracy is lower bounded by 0\% as opposed to 50\%, because we always transfer a sentence to the opposite sentiment. The lower-bound, zero transfer accuracy, is achieved by a trivial model that copies the input.} resulting in inflated cosine similarity and word overlap scores. We also examined the number of times each model generates exact copies of the source sentences during style transfer. We notice that the style-embedding model reconstructs exactly the same sentence in style transfer $24\%$ of the time, whereas all other models is lower than $6\%$. Therefore, we do not think the style embedding an effective model for text style transfer.

The other two competing methods~\cite{shen2017style,zhao2018adversarially} achieve reasonable transfer accuracy and cosine similarity. However, our model significantly outperforms them by 10\% transfer accuracy as well as content preserving scores (measured by cosine similarity and the word overlap rate). This shows our model is able to generate high-quality style transferred sentences, which in turn indicates that the  latent space  is well disentangled into style and content subspaces. 

Regarding language fluency, we see the VAE is better than DAE in both experiments. This is expected as VAE regularizes the latent space by imposing some probabilistic distribution. We also see that our method achieves significantly lower , showing that 
Our VAE model also produces the most fluent sentences for both tasks, which is corroborated by the manual evaluation results.

Table~\ref{tab:ablation-results} presents the results of ablation tests on the Yelp dataset.
We see that both the style adversarial loss and multi-task classification loss play a role in the strength of style-transfer, and that they can be combined to further improve performance.



The manual evaluation results (Table \ref{tab:manual-evaluation}) show that our VAE model attains the best scores for transfer strength and generated sentence quality, as well as the second best score for content preservation amongst all the evaluated models.

Some examples of style-transfer sentence generation are presented in Table~\ref{tab:transfer-samples}.
We see that, with the empirically estimated style vector, we can reliably control the sentiment of generated sentences.

\section{Conclusion and Future Work}
In this paper, we propose a simple yet effective approach for disentangling the latent space of neural networks using multi-task and adversarial objectives.
Our learned disentanglement approach can be applied to text style-transfer tasks.
It achieves similar content preservation scores, and significantly better style-transfer strength and language fluency scores compared to previous state-of-the-art work.

For future work, we intend to evaluate the effects of disentangling the style space for datasets with greater than two distinct styles.
We would also like to explore the possibility of aligning each encoded style distribution to a unique prior, which could be sampled from at inference time for style-transfer, as opposed to using the empirical mean of training-time style embeddings.

%\section{Acknowledgements}
%We acknowledge the support of the Natural Sciences and Engineering Research Council of Canada (NSERC) [261439-2013-RGPIN], and Amazon Research Award.
%The Titan Xp GPU used for this research was donated by the NVIDIA Corporation.


\begin{table*}[ht]
	\centering
	\begin{tabular}{| l || c | c | c | c |}
		\hline
		\tabc{2}{Objectives}                                                            & \tabh{Transfer} & \tabh{Cosine}     & \tabh{Word}    & \tabh{Language} \\
		                                                                                & \tabh{Strength} & \tabh{Similarity} & \tabh{Overlap} & \tabh{Fluency}  \\
		\hline
		\hline
		$\loss{AE}$                                                                     & 0.106           & 0.939             & 0.472          & -12.58          \\
		\hline
		$\loss{AE}$, $\loss{mul(s)}$                                                    & 0.767           & 0.911             & 0.331          & -12.17          \\
		\hline
		$\loss{AE}$, $\loss{mul(c)}$                                                    & 0.155           & 0.931             & 0.626          & -11.21          \\
		\hline
		$\loss{AE}$, $\loss{adv(s)}$                                                    & 0.782           & 0.886             & 0.230          & -12.03          \\
		\hline
		$\loss{AE}$, $\loss{adv(c)}$                                                    & 0.103           & 0.940             & 0.470          & -12.92          \\
		\hline
		$\loss{AE}$, $\loss{mul(s)}$, $\loss{mul(c)}$                                   & 0.822           & 0.928             & 0.561          & -11.32          \\
		\hline
		$\loss{AE}$, $\loss{mul(s)}$, $\loss{adv(s)}$                                   & 0.912           & 0.866             & 0.171          & -9.59           \\
		\hline
		$\loss{AE}$, $\loss{mul(s)}$, $\loss{adv(c)}$                                   & 0.781           & 0.906             & 0.318          & -11.99          \\
		\hline
		$\loss{AE}$, $\loss{mul(c)}$, $\loss{adv(s)}$                                   & 0.825           & 0.894             & 0.507          & -10.94          \\
		\hline
		$\loss{AE}$, $\loss{mul(c)}$, $\loss{adv(c)}$                                   & 0.149           & 0.929             & 0.621          & -11.22          \\
		\hline
		$\loss{AE}$, $\loss{adv(s)}$, $\loss{adv(c)}$                                   & 0.780           & 0.884             & 0.214          & -12.03          \\
		\hline
		$\loss{AE}$, $\loss{mul(s)}$, $\loss{mul(c)}$, $\loss{adv(s)}$                  & 0.922           & 0.903             & 0.476          & -9.65           \\
		\hline
		$\loss{AE}$, $\loss{mul(s)}$, $\loss{mul(c)}$, $\loss{adv(c)}$                  & 0.844           & 0.925             & 0.556          & -11.13          \\
		\hline
		$\loss{AE}$, $\loss{mul(s)}$, $\loss{adv(s)}$, $\loss{adv(c)}$                  & 0.934           & 0.864             & 0.157          & -10.11          \\
		\hline
		$\loss{AE}$, $\loss{mul(c)}$, $\loss{adv(s)}$, $\loss{adv(c)}$                  & 0.806           & 0.896             & 0.500          & -9.98           \\
		\hline
		$\loss{AE}$, $\loss{mul(s)}$, $\loss{mul(c)}$, $\loss{adv(s)}$, $\loss{adv(c)}$ & 0.934           & 0.904             & 0.473          & -9.84           \\
		\hline
	\end{tabular}
	\caption{Ablation tests.}
	\label{tab:ablation-results}
\end{table*}

\bibliography{main}
\bibliographystyle{aaai}

\end{document}
